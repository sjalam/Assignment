\documentclass[12pt]{article}
\title{Gender Discrimination Curbs the Opportunity of Women to Excel}
\author{SK JANE ALAM\\sjalam088@gmail.com}
\date{}
\begin{document}
\maketitle
\section{ABSTRACT}
Gender Discrimination has always been a hot topic in every discussion. The primary focus in
economics has always been the growth and development of the economy. However we often
tend to neglect seeing the development of women which implicitly affects holistic growth.
Gender discrimination being a social issue originates from certain microlevel beliefs. The
stereotype belief on women possessed by households, society, employers often questions the
abilities of a woman in executing certain tasks. Difference in traits between men and women
creates barriers to women in every stage of their career influencing their decision making.
Superimposition of do’s and don’t by society snatches the freedom of women to make their
own choices.
\section{INTRODUCTION}
Despite various campaigns on feminism and various policies like “Beti bacchao..beti
padhao”, women are still struggling to build a significant place for them in this son-preferred
society even in the 21st century. These biases among men and women have worsened so much
that right after birth, gender largely determines the fate and future lifestyle of a kid. Women
have always been victims to this discrimination through the ages and continue to face the
same even today in different forms. Both men and women have their equal share of human
rights. But still women are deprived of their rights and freedom.
Keeping this ageless issue the base of our term paper, we will discuss what all are the
behavioural traits at the societal level that still keep the discrimination strong and intact
despite the monumental endeavours taken by government, NGOs, various other women
centric organisations and many self-help groups in empowering women. A very peculiar
behaviour that drew our attention and motivated us to write this paper is there is a significant
gap between women enrolment in higher education and their participation in the labour
market. The question that arises is why are women so unlikely to work despite having both
capabilities and opportunities? In some under-developed regions we often witness an
intergenerational conversation between the women of the families addressing the younger
one to be the privileged one for getting all the basic treatments, especially a quality education
at an ease for which the elders had to fight for. Intra-house discrimination among kids on the
basis of gender has reduced a lot in the middle and the upper sections of society. Our concern
in this paper are the girls of these sections who are provided with all the basic amenities like
quality education and healthcare in an adequate way but still get underrepresented in certain
job profiles. A very low participation of women has been seen in STEM (Science,
technology, engineering, and management) fields. For example, though women
representation in the physical sciences improved during the time period of 2001-2010,
women still account for only 22 percent of these disciplines in 2010 (Armstrong and
Jovanovic, 2015; NSF, 2013). Even in medical disciplines, where, as of 2018, the number of
women enrolled in medical schools exceeded men for the first time, there is still a persistent
underrepresentation of women at senior academic or leadership positions that we cannot
explain by a time lag between degree completion and career trajectory. Therefore we go
further on the issue of why women are more underrepresented in some scientific,
engineering, and medical disciplines than others.

\section{METHODOLOGY}
This study is descriptive in nature and tries to find out the beliefs at the micro level that cause
hindrances in women's development. As a research method, emphasis is given on content
analysis and it takes a form of descriptive research. Most of our term paper consists of
Literature Reviews. Moreover we try to focus on both the household perspective and the
employer’s perspective in order to find how there exists an employment gap based on gender.

\section{MOTIVATION}
Women’s decisions to not enter the labour market or to drop out of it at an early age despite
acquiring the skills and education to sustain oneself in middle and high income jobs, are
influenced to a huge extent by the prevailing social attitudes and practices as well as the lack
of economic incentives and policy interventions to help bring about gender equality. This
motivated us to study the entrenched structural causes of gender inequalities and develop an
understanding of the patriarchal power at play in order to systematically address the problem.
We suspect that a number of behavioural traits are behind this issue, which include but are not
limited to the following: Stereotypes- Certain qualities, some seemingly benign (like “women
are nurturing”) and some overtly hostile (like “women are irrational”) are perceived to be
associated particularly with women, so women are discouraged from entering the labour
force and instead are expected to take care of the family and employ more time and effort to
household work compared to men. Information avoidance- People weigh and interpret
evidence in a fashion that supports what they are motivated to believe about what a woman
ought to do with her life. For example, even if they have evidence that women are excelling
in a certain male-dominant career field, they might still think it’s a bad idea for a woman to
pursue that career path. This happens because they are not focussing their attention on this
information and are prone to denigrate the quality of evidence that contradicts their beliefs.
Motivated reasoning and cognitive dissonance- Models of motivated reasoning state that
people choose not only their actions but also their beliefs, since their overall utility is higher
when there is greater congruence between actions and beliefs. In our context, it might happen
that women are not participating in the job-market or settling for jobs that are below their
capacities; but they are choosing their beliefs in a way that reinforces their choice of actions
and hence there is little or no belief-updating.

\section{LITERATURE REVIEW}
\subsection{BARRIERS FACED AT THE HOUSEHOLD LEVEL
}
Whenever the question is of the well being of the family, it is women who need to sacrifice or
compromise her choices. This is not a new practice, since the later vedic age it was seen that
women’s jewelries were the last resort to save her family from the financial crisis. However
this concept still holds in this generation too, with the addition of giving up a job for the
benefit of the family. Therefore it is not surprising to witness that even after completing more
years of schooling than men, women make educational choices that translate into lower
expected labor market earnings. However, If women are lagging behind men not because of
labor market discrimination but because of the different choices they make, it is key to
understand why they are making such different choices. According to, Bronstein and
Farnsworth, 1998; MacNell et al., 2014; Milkman et al., 2015; Moss-Racusin et al., 2012;
Settles et al., 2006; Urry, 2015 women may experience implicit bias and structural barriers at
every career stage, including at critical junctures such as consideration for graduate school
admission, recruitment into a laboratory for graduate research, consideration for postdoctoral
positions, recruitment to fill tenure-track faculty positions, and evaluation for promotion in
rank.

\subsection{Household beliefs}
These biases arise from Gender Stereotypes. Economics mainly views gender stereotypes as
a manifestation of statistical discrimination (Phelps 1972, Arrow 1973, Aigner and Cain1977)
In statistical discrimination models, men and women are treated differently due to imperfect
information. Gender stereotypes in these models are rational beliefs about a group member (a
woman or a man) based on the aggregate distribution of a trait or skill in the gender group.
They are beliefs, shared by men and women, about what men and women should or ought to
do. Unlike, social psychology stresses that gender stereotypes are not only descriptive but
also prescriptive. According to Akerlof and Kranton(2000), this prescriptive nature of Gender
Stereotype results in gender identity norms where men and women modify their beliefs on
their capabilities based on their gender. Moreover, the prescriptive nature of gender
stereotypes also naturally derives from a shared education, with views about gender roles and
gender skills passed on from parents to their male and female offspring. Parents, explicitly or
implicitly, “gender talk” to their children by emphasizing gender categories and teaching
what are appropriate and inappropriate behaviors for boys and girls (Endendijk et al. 2014).
A paper by Brenøe 2018 shows that this transmission of stereotypes about gender roles
appears more prominent in families with mixed-sex children, suggesting that having an
opposite-sex sibling increases exposure to gender- stereotypical socialization in the home.
The prescriptive nature of gender stereotypes explains why women provide more unpaid care.
The gender identity norms compel women to think they are good at caring and nursing tasks.
As stated by Babcock,Recalde, and Vesterlund 2017, providing this care or engaging in those
tasks might still be utility maximizing if it allows the individual to comply with the gender
identity norms: “being a housewife is just as fulfilling as working for pay” or “a working
mother can establish just as warm and secure a relationship with her children as a mother who
does not work.”
Employment of women can further magnify unequal gender relations by reinforcing dutiful
roles of a wife or a daughter, the control of women’s earnings by male relatives, or continued
social disapproval and violence against women who do not behave in accordance with
patriarchal norms. These create disincentives for women to participate in the labour
force.Again, Amartya Sen came up with the capability approach which argues that the ability
for policies to improve well-being should be evaluated not just on the basis of “functionings”
(achieved status, e.g., material standard of living, having a good job, being healthy) but also
on the basis of “capabilities” (i.e., opportunities for such achievements). People should have
the freedom to choose from many different combinations of “functionings.” The fact that
people do not perceive that they lack that freedom, or the fact that they adapt to misery, does
not mean that there is no need for corrective policy, even if revealed preferences do not seem
to call for such correction. As powerfully stated in Nussbaum(2001, p. 42),
“Even when women appear to be satisfied with such customs, we should probe more
deeply. If someone who has no property rights under the law, who has had no formal
education, who has no legal right of divorce, who will very likely be beaten if she
seeks employment outside the home, says that she endorses traditions of modesty,
purity, and self-abnegation, it is not clear that we should consider this the last word on
the matter. . . . Women’s development groups typically encounter resistance
initially,because women are afraid that change will make things worse.”


\subsection{Barriers in the educational choice and choice of career}
Women at different stages of their lives face barriers that keep on updating their beliefs and
their choices. With the support of the gender identity norm the very first barrier they face is in
choosing their educational choices. The stereotype that women are poorer in maths
discourages them to opt for STEM fields. As a result there has always been
underrepresentation of women in these fields, adding to this is the conventional belief of the
family members. Career in STEM fields always showed a very low representation of women.
Their belief of task over there being herculean, presence of sexual harrasments and
discrimination, irregularity in shifts discourage them to pursue.


\section{BARRIERS CAUSED BY FIRMS}
Traits such as assertiveness, confidence, boldness, risk-taking, independence, and
self-promotion are valued, rewarded, and seen as standards while hiring (Diekman and
Steinberg, 2013). Stereotypically “masculine traits” (e.g., assertiveness, ambition, and
competitiveness) and “feminine traits” (e.g., warmth, supportiveness, and collegiality) are
exhibited by both women and men. However, differences in these traits lead to gender bias in
the labour market, which takes the following forms.

\subsection{Recruitment Bias}
Employer’s belief of men and women having different traits is reflected while recruiting
employees. For example, an experimental study was held by Carli et al. (2016) where
participants were asked to list traits they associate with scientists and with men and women
irrespective of profession, the traits identified for scientists and men overlapped to a greater
extent than did the traits identified for scientists and women. In support of this finding,
(Banchefsky et al., 2016) stated nonscientists are less likely to believe a woman is a scientist
if she has a feminine (rather than masculine) appearance. This biased belief results in a
connection of masculinity with jobs in STEM fields. Another study by Sarsons et.al (2019)
talks about the primary consideration of a physician's choice is the surgeon quality. However
this quality is often influenced by gender identity. The finding says that physicians use
information about individual female, and not male, surgeons to update their beliefs about
other female surgeons in the same specialty. However, it has been noticed that a physician's
gender doesn't matter while making this choice. In addition, physicians use their experience
with one woman to infer the ability of other female surgeons. After a bad experience with one
female surgeon, physicians become less likely to refer to other female surgeons in the same
specialty.
\subsection{Discrimination in wages}
Even today, there is a considerable gender gap in pay that would decrease and might become
nil if some firms did not have an incentive to disproportionately reward individuals who
worked long hours and worked particular hours. Most of the studies involving the gender
wage gap produced estimates of an “explained” and a “residual” portion. The “residual” is
seen as “wage discrimination” as it is defined by the difference in earnings between
observationally identical males and females. The explained portion of the gender wage gap
has decreased over time because human capital investments like years of education and
labour market experience between men and women have converged. However, the residual
portion of the gap increased compared to the explained portion. An explanation for this
increase by Claudia Goldin (2014) involves an application of personnel economics that
relies on labor market equilibrium with compensating differentials and endogenous job
design.
The pay gap depends on how firms reward individuals who have different levels of desire for
amenities that are various aspects of workplace flexibility. This concept incorporates the
number of hours that have to be worked as well as the particular hours worked, being “on
call,” providing “facetime,” being around for clients, group meetings etc.
Focusing on certain occupations provides further thoughts on how to equalize earnings by
gender. Some occupations exhibit linearity of earnings with respect to time worked while
others exhibit nonlinearity (convexity). Research shows that the gender gap is high when
there is nonlinearity and lower in case of linearity. This happens because of differences in job
flexibility and continuity across occupations. Having young children (less than two years old)
reduces participation for all college graduate women. Certain occupations impose heavy
penalties on employees who want fewer hours and more flexible employment. The lower
remuneration might lead to shifts to an entirely different occupation or to a different position
within an occupational hierarchy or dropping out of the labor force altogether. Occupations
like business and law show a nonlinear (convex) relationship of total earnings with respect to
hours or to the flexibility of hours. Corporate and financial sectors also impose heavy
penalties on deviation from the norm. In comparison with these occupations, people in
technology and science have far more time flexibility, fewer client and worker contacts,
fewer working relationships with others, more independence in determining tasks, and more
specific projects with less discretion over them. Each of these characteristics produce a more
linear relationship between hours and earnings and the greater linearity produces a lower
residual difference in earnings by sex.

\section{ CONCLUSION}
Over the past fifty years, there has been steady progress with respect to efforts that
mainstream gender in standard economic decision-making. Attention to gender equality and
empowerment of women is important for implementing effective development policies.
Various initiatives have resulted in the development
of new ways to measure economic activity which include household systems of production
and distribution. Policies are being implemented that enhance the health and education of
women and promote women’s labor force participation. However, still a lot needs to be done.
The efficacy of macroeconomic policies should be measured in such a way that takes into
account the unpaid economic activities and labor in the household sector. Development
processes should incorporate a framework for reallocating resources, providing socialized
support for care and promoting the equal sharing of responsibilities between women and men.
Having a vision for change along with a will to make it happen can surely bring about social
transformation.
%\ref{References}

%https://www.india-briefing.com/news/women-and-work-in-india-trends-and-analysis-24758.html/#:~:text=According%20to%20annual%20bulletin%20of,merely%2022.2%20percent%20for%20females



\end{document}